\begin{document}
\selectlanguage{english} 
\begin{abstract}
In this paper, the basis for Global Navigation Satellite Systems (GNSS) is presented and a comparison between the expected precision of the relative position between two receivers is presented based on observations made by two stationary receivers. The positioning is compared between the solution from two algorithms. The first relative estimate is based on making individual position estimates for the receivers and calculating the difference. The second uses a so called double difference method for estimating the relative position. The assumption is that systematic noises will be more successfully mitigated by the use of a double difference algorithm, which is also verified through simulations. The result of the observations is that the double difference performs slightly better, with a measured mean error of 4.8 and 4.9 m, compared to the relative position of the global estimates of 5.6 and 5 meters. These errors indicates that the random and unmodeled noises were larger in the sampling series than what was expected. A continued work should implement a filter that can lower the noise levels in the observations.
\end{abstract}
\selectlanguage{swedish} 
\begin{abstract}
I detta dokument presenteras grunderna för Satellitnavigation (GNSS) och en jämförelse mellan den förväntade precisionen hos den relativa positionen mellan två GNSS-mottagare baserat på mätningar från två stationära mottagare. Positioneringen jämförs för lösning som erhålls som differensen mellan individuella globala positioner samt när en differentierad positionslösning implementeras. Antagandet till grund för undersökningarna är att den differentierade estimatorn är bättre på att minska effekterna av systematiska brus. Detta verifieras även med hjälp av simulering. Den positionslösning som erhålls visar att positionen baserat uteslutande på Satellitnavigation kan förväntas ligga på strax över 5 m, med uppmätta medelfel på 5 och 5,6 m, samt strax under 5 m för den dubbeldifferentierade estimatorn med uppmätta medelfel på 4,8 samt 4,9 m. Magnituden på felen indikerar att de omodellerade brusnivåerna var större än väntat. Ett fortsatt arbete bör söka att utveckla lösningen och implementera ett filter som kan minska brusnivåerna.
\end{abstract}
\selectlanguage{english}

\newpage
I would like to express my deepest gratitude to my supervisors Håkan Carlsson and Linnea Persson who I believe have gone well beyond their expected effort and invested a lot of time in aiding me in discussing the theory and investigating the implementation. I would also like to express a thank you to family and friends who have supported me through this project.
\end{document}