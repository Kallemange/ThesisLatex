
\begin{document}
\chapter{Method}
\section{Simulation of data}
Simulations of data are performed in order to verify the theoretical behaviour of the models under the influence of different noise levels. The theoretical behaviour is then compared to the observations using different levels of bias.
\subsection{Testing different error sources impact on global position}
The methods of global positioning and double difference's behaviour are also tested using simulated data to verify the theoretical behaviour of the estimator compared to that of the actual measurements. The simulations are based on creating pseudorange measurements between a stationary point on earth and corrupting it with noise, where actual receiver positions are used, and satellites are positioned according to the orbits from the data contained in the ephemeris messages. 
\par 
The simulations cover the error sources mentioned in section \ref{signalAndError} except for the atmospheric noise. From this, a theoretical variance can be derived based on the different error sources of the real sample series and identify potential errors in the estimate method. 

\section{RMSE of relative position from global position and DD estimate }\label{RMSE}
Since the DD-method has been shown to improve the relative estimate, mentioned in section \ref{previousResearch}, the assumption is made that the DD-method is superior to that of two individual global fixes in mitigating the effect of a bias. The behaviour is simulated using increasing levels of bias. This means that any two simulated observation between a satellite and the receivers will be of the form 
\begin{align}
y_a&=||{\bf p}^{(i)}-{\bf p}_{a}||+c\Delta t_a+\eta^{(i)}+ \epsilon_a\\
y_b&=||{\bf p}^{(i)}-{\bf p}_{b}||+c\Delta t_b+\eta^{(i)}+\epsilon_b.
\end{align}
The notation is consistent with that above. In the simulations, the shared non-white noise $\eta^{(i)}$ is randomly sampled and will be constant per satellite for the observation series. The simulations are then performed using increasing magnitudes for the bias level. The result is then presented as the root mean square error (RMSE) of the estimate, defined as
\begin{align}
e_{RMS}	&=\sqrt{\sum   |\bf d-\hat{d}|/n} \nonumber \\
		&=\sqrt{\frac{1}{n}\sum_{i=k}^{n} ({\bf d-\hat{d}}[k])\cdot ({\bf d-\hat{d}}[k])^T} \label{eqERMS}
\end{align} 
for a true baseline vector $\bf d$ and the corresponding estimated distance $\bf \hat{d}[k]$ for an epoch $k$. Since the experiment is conducted using the north and east direction separation, the baseline vector will be set to respectively $[10, 0,0]$ m and $[0, 10, 0]$ m.


\section{GNSS positioning}

\subsection{Estimating satellite position}
To verify that the satellite trajectories are correct the satellite positions are calculated for a given time span using the method described in section \ref{chap:ephPositioning} from the received ephemeris data. This is then compared to the historical satellite positions available on-line\footnote{e.g. \url{https://in-the-sky.org/} and \url{https://www.gnssplanning.com/\# /charts}}. Plots showing the trajectory, as well as elevation and azimuth for a given time frame are produced and compared in section \ref{satelliteTrajectory}. 

\subsection{Data extraction from sensor}
The INS unit allows for data sampling and streaming in real-time as well as logging for post-processing through three different types of software. A GUI called EvalTool is available from the producer Inertial Sense\footnote{\url{https://docs.inertialsense.com/user-manual/software/evaltool/}} for logging data for most applications, both fused and unfused data from the GNSS-receiver and the IMU units. In addition to that, there is a command-line tool called CL Tool for logging of much the same functionality\footnote{\url{https://docs.inertialsense.com/user-manual/software/cltool/}}. However, for the sake of this project, unprocessed pseudorange observation data from the receivers were required to implement and compare the single and double-difference methods described in chapter \ref{RelPos}. In order to extract those, data must be parsed directly from the Software Development Kit (SDK) projects available. A logger, producing comma-separated values (.csv)-type log files of the received packages is available at \url{https://github.com/Kallemange/Communications} for post-processing. More information on the logger and data structures in use can be found in appendix \ref{structsAndLogs}.
\subsection{Experimental setup}\label{experimentalSetup}
In order to test the receiver's behavior over time, two receivers are placed stationary at a baseline of 10 m pointing first in N-direction as well as in E-direction, with measurements taken for approximately 30 minutes in Uggleviksk\"allan, a glade in the forest on coordinates: 59.353$^o$N, 18.073$^o$E, shown in figure \ref{fig:Uggleviken}. One receiver was placed close to the pin indicated in the figure, and the other positioned east and south of it.
\begin{figure}[!h]
\includegraphics[scale=0.4]{Method/Uggleviken}
\caption{\label{fig:Uggleviken} Uggleviken, place where observations were made. Image taken from Google Maps: \url{https://www.google.com/maps}}
\end{figure} 
The directions were set using a 
digital compass on an android phone and the distance through using a measuring tape. The logger is started for the receivers separately, but are connected to the same computer where the log files are stored. 

\subsection{Global and relative positions from onboard estimate}
The processed estimates from the onboard electronics are sampled and logged in parallel to the raw data. This contains information on the global position in an ECEF or LLA-frame, HDOP and VDOP values as well as sampling time. This data will be called the onboard estimate. From this, an estimate of the variance in each direction can be made to reference that of the raw data as well as a relative estimate of the positions.
\subsection{Global and relative position estimates from raw data}
The method for how the positioning is made in the individual as well as the relative case using the log files is presented in the following section. The solutions have only been implemented based on log files and are not made to run real-time.
The solution is calculated in two steps: 
\begin{enumerate}
\item Load the data from log files into an array with a struct for each epoch. 
\item Calculate the solution per epoch from the observation and ephemeris data
\end{enumerate} 
\subsubsection{Global positioning}
The positioning of each receiver only utilizes the ephemeris data collected by the same receiver and only observation data that has a corresponding ephemeris reading is used. The method for positioning which is implemented follows the description in section \ref{stateEst}. The solution is an instantaneous estimate for each epoch, indicating that the previous estimate is not taken into account for the current one. This will produce a solution calculated in an ECEF coordinate frame, which is then projected to a NED frame. The solution includes a global position, calculated as described in section \ref{globalWEstimator} with the weighted estimator of equation \ref{SNRWeights}, an estimate of the HDOP and VDOP values, as described in (\ref{eq:HDOP}-\ref{eq:VDOP}) as well as a variance over the solution, calculated per direction in a NED-frame.

\subsection{DD relative positioning}
For the double difference relative positioning algorithm, not only must the observation data for each epoch match that of the ephemeris data, but also must be equal between receivers. For each epoch, any data not contained in both is discarded. The position must always be calculated using one of the receivers, which will be called $r_a$, as reference, and the other, called r$_b$ fitted to it. This method, which follows the instructions for double difference in section \ref{RelPos} utilizes that clock error cancels out and will not estimate either. The satellite position is instead calculated at the nominal time of observation $t_{rec}$. This is due to the angular change, as opposed to the position change, between satellite and receiver is negligible within the time frame of a sample. 
\par
The relative position estimate also requires the receiver position ${\bf p}_a$ in order to calculate the unit vector ${\bf e}^i$ pointing to a satellite from a receiver. The position used is that given by the onboard estimate. The system of equations is then solved for the given reference satellite, which will be selected as that with the highest SNR value for each epoch, as suggested in \cite{BLUE}. The solution will give an instantaneous relative positional estimate for r$_b$ with regards to r$_a$ in an ECEF frame, which is then projected down to a NED solution through the point given. This also implies that a global position is never calculated through this method. Given that the receivers were stationary, the estimates are expressed as a mean and a standard deviation in each direction.

\end{document}
